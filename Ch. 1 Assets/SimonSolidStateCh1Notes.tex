%\def\standalonechapter{} % comment or uncomment for 

\ifdefined\standalonechapter
\documentclass{article}
% ==== Preamble ====
\usepackage[english]{babel}
\usepackage{outlines}
\usepackage[letterpaper,top=2cm,bottom=2cm,left=3cm,right=3cm]{geometry}
\usepackage{amsmath,amssymb}
\usepackage{graphicx}
\usepackage[colorlinks=true, allcolors=blue]{hyperref}
\usepackage{fancyhdr}

% ==== Custom Macros ====
\newcommand{\boldvec}[1]{\vec{\mathbf{#1}}}

\newcommand{\defn}{\textbf{Definition: }}

\newcommand{\chaptertitle}[2]{%
    \clearpage

    % Reset section counters for new chapter
    \setcounter{section}{0}
    \setcounter{subsection}{0}

    % Clear header marks so old sections don't bleed through
    \markboth{#1}{}

    % Chapter title page
    \thispagestyle{empty}
    \begin{center}
        \vspace*{\fill}
        {\LARGE \textbf{#1}}\\[1em]
        {\large #2}
        \vspace*{\fill}
    \end{center}
    \clearpage
}


% ==== Setting Counters ====

\begin{document}
\fi

\setcounter{subsection}{0}

\pagestyle{fancy}
\fancyhead[L]{Ch. 1: About Condensed Matter Physics}   % Chapter name
\fancyhead[R]{\rightmark}  % Section name %

\setcounter{section}{1}
\subsection{What Is Condensed Matter Physics}
\begin{outline}[enumerate]
    \1 Author quotes wikipedia; however, CMP is best understood as the study of how macroscopic properties emerge form microscopic interactions within large systems.
    \1 Extremely large sub-field of physics, where materials science, quantum mechanics, and abstract mathematics come together!
\end{outline}

\newpage
\subsection{Why Do We Study Condensed Matter Physics?}
\begin{outline}[enumerate]
    \1 Condensed matter describes most of the materials that we regularly interact with. As such, it is very useful and extremely useful. Think: "As wide as an ocean, as deep as... an ocean."
    \1 It tells you how properties of materials emerge.
    \1 The single best laboratory for the study of quantum and statistical physics. 
\end{outline}

\newpage
\subsection{Why Solid State Physics?}
\begin{outline}[enumerate]
    \1 Solid state physics is the largest sub-field of condensed matter physics that studies ordered solids. 
    \1 Technologically useful, as many technological innovations are touched by solid state researchers.
    \1 Gateway to the further study of physics.
\end{outline}

\newpage 

\section*{Key Concepts and Equations from Chapter 1}

\subsection*{All Subchapters} 
\begin{outline}[enumerate]
    \1 Solid state physics is the largest sub-field of condensed matter physics that studies phenomena in ordered solids. It fuses together abstract mathematics, quantum mechanics, and statistical physics.
    \1 Solid state physics is also highly applicable to modern technology, and is a gateway to the study of more advanced concepts in physics.
\end{outline}

% ===== Chapter content ends here =====

\ifdefined\standalonechapter
\end{document}
\fi