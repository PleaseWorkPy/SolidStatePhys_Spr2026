\documentclass{article}

\usepackage[english]{babel}
\usepackage{outlines}
\usepackage[letterpaper,top=2cm,bottom=2cm,left=3cm,right=3cm,marginparwidth=1.75cm]{geometry}
\usepackage{amsmath}
\usepackage{graphicx}
\usepackage{amssymb}
\usepackage[colorlinks=true, allcolors=blue]{hyperref}
\usepackage{fancyhdr}

% ==== Custom Macros ==== %
\newcommand{\boldvec}[1]{\vec{\mathbf{#1}}} % For a vector quantity
\newcommand{\boldvecdot}[1]{\dot{\vec{\mathbf{#1}}}} % For the first derivative of a vector quantity
\newcommand{\boldvecddot}[1]{\ddot{\vec{\mathbf{#1}}}} % For the second derivative of a vector quantity
\newcommand{\defn}[0]{\textbf{Definition: }}
% ======================= %

\title{Chapter 3: Electrons in Metals: Drude Theory}
\author{Alex Hardie}


\begin{document}


\vspace*{\fill}       % push down to vertical center
\begin{center}
\stepcounter{section}
    \section*{Chapter 4: Sommerfeld (Free Electron) Theory}
    \addcontentsline{toc}{section}{Title Page}
\end{center}
\vspace*{\fill}       % push up to vertical center


\newpage

\pagestyle{fancy}
\fancyhead[L]{Ch. 4: Sommerfeld (Free Electron) Theory}   % Chapter name
\fancyhead[R]{\rightmark}  % Section name %

\subsection*{Intro}
\begin{outline}[enumerate]
    \1 
\end{outline}

\newpage
\setcounter{section}{4}
\subsection{Basic Fermi--Dirac Statistics}
\begin{outline}[enumerate]
    \1 
\end{outline}

\newpage
\subsection{Electronic Heat Capacity}
\begin{outline}[enumerate]
    \1
\end{outline}

\newpage
\subsection{Magnetic Spin Susceptibility (Pauli Paramagnetism)}
\begin{outline}[enumerate]
    \1
\end{outline}

\newpage
\subsection{Why Drude Theory Works So Well}
\begin{outline}[enumerate]
    \1
\end{outline}

\newpage
\subsection{Shortcomings of the Free Electron Model}
\begin{outline}[enumerate]
    \1
\end{outline}

\newpage

\section*{Key Concepts and Equations from Chapter 4}

\subsection*{Ch. 4 Introduction} 
\subsection*{4.1} 
\subsubsection*{4.2}
\subsubsection*{4.3}
\subsubsection*{4.4}
\subsubsection*{4.5}


\newpage

\section*{Suggested Further Readings for Clarification} 
Ashcroft and Mermin [INSERT] \\
Kittel [INSERT] \\
YOUTUBE VIDEO \\
MATH/PHYS STACK EXCHANGE \\

\end{document}